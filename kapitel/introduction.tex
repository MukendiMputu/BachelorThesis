% introduction.tex
\chapter{Introduction}\label{ch:intro}

One of the most challenging tasks in software development is developing a long-term strategy for creating, managing and updating the documentation for the end users as the software product develops, adapts or increases in complexity. The challenge lies in the fact that the development team tasked to document the software has to do it manually by first selecting the information with the most value to the end users, structuring it and then when required updating it. In other words, the developer has to reproduce the steps or whole scenarios a potential end user would go trough and document where useful information are to be provided in order to reduce the time needed to understand the main functionality of the software. 

Completing this process once does not necessarily free the developer from the task, it is a cycle that has to be repeated every time an update is introduced or a relevant part of the program has been changed. For a small static web page this might not sound dramatic, but considering complex applications developed by multiple teams, this task can raise the cost in time and resources as the information about the whole project has to be collected for the design of the documentation\cite{IEEE}. So the problem is to find a way to automatically go through all those steps and generate parts of the model extended with semantic description and later on assemble those to a complete documentation.

Software projects are bound by budget and deadline requirements. This leads to the fact that automation processes become essential where the reduction of the development time can be increased. It is therefore of significant advantage for the developing team \textit{and} the end user that the generation of user documentation is modeled in such a way that it always reflects the most current development status of the software product. This thesis introduces the use of a graphical \acrfull{dsl} as an alternative way of modeling end user documentation.

For simplicity's sake, this thesis proposes a solution in which one first models the user sequences using graphical DSLs, and then uses generators to generate a modeling platform which then allows the designer to model the sequences to be automatically replayed in the web application.

\section{Objectives}\label{sec:objectives}

This thesis proposes a solution for automatically generating end-user documentation for web application by means of a graphical \acrfull{dsl}. We developed an applications for modeling different user action sequences in a web application using the \textsc{Cinco} SCCE Meta Tooling Framework~\cite{Cinco} and subsequently generate an end-user documentation in markdown syntax. The \textsc{Cinco} framework is a generator-driven development environment for domain-specific graphical modeling tools. It is based on the Eclipse Modeling Framework and Graphiti Graphical Tooling Infrastructure, but aims to hide much of their complexity and intricate \acrshortpl{api}\footnote{More information can be found at \url{https://cinco.scce.info/}}. For evaluation purposes, we document a task management web application of our own. Nonetheless, the applied method can be extrapolated to any other web application, since the elementary building blocks of those applications are the same.

The project has a Maven nature, which allows us to manage the package dependencies and take advantage of its build life cycles for building and distributing the application\footnote{The code for the application can be found here \url{https://github.com/MukendiMputu/UserDocGenerator}}.

The markdown-based documentation is supplemented with screenshots taken with the Selenium-WebDriver using the \gls{Selenium} \acrfull{pom} development pattern to replicate user actions. In addition to that we configure \gls{VuePress} to serve the generated markdown files containing references to the screenshots as static site.

\section{Outline}\label{sec:outline}

\improvement[inline]{The following section needs to be rewritten!}

Chapter \ref{ch:DSL} presents the \textsc{Cinco} SCCE Meta Tooling Framework, in particular the two metalanguages (\acrfull{mgl} and \acrfull{msl}) that constitute the specification (Section \ref{sec:MGL} and \ref{sec:MSL}) and also the \acrfull{cpd} (Section \ref{sec:CPD}). The chapter then rounds off with an outline of the Generator classes written in Xtend, a Java dialect that offers much flexibility and expressiveness in its syntax (Section \ref{sec:GEN}).\\The specifics of the generated modeling platform are described in chapter \ref{ch:CP}. Whereby Section \ref{sec:gDSL} explains the basic concepts of a graphical domain-specific language, then Sections \ref{sec:FeatModElem}, \ref{sec:DocModElem} and \ref{sec:GenProcess} relate about different modeling elements for configuring and designing the end user documentation.\\Chapter \ref{ch:userDoc} talks about the composition of user documentation as recommended by ISO/IEC standards. In chapter \ref{ch:eval} a in-depth explanation of the methodology is given. Beginning with the evaluation of the selected work method in Section \ref{sec:meth} followed by the system setups and the results (Section \ref{sec:setup} and \ref{sec:res}) and finally, in Section \ref{sec:disc} concluding with a discussion on the approach taken in this thesis.\\
Further improvements of the program are presented in chapter \ref{ch:futwork}. Chapter \ref{ch:relwork} relates about similar work on this topic. Lastly, chapter \ref{ch:concl} winds up with a discussion on the thesis.