% introduction.tex
\chapter{Introduction}
\section{Motivation and Background}
One of the most challenging tasks in software development is developing a long-term strategy for creating, managing and updating the documentation for the end users as the software product develops, adapts or increases in complexity. The challenge lies in the fact that the developer team, consisting of one or more persons, tasked to document the software has to do it manually by first selecting the information with the most value to the end users, structuring it and then when required updating it.

In other words, the developer has to reproduce the steps or whole scenarios a potential end user would go trough and document where useful information are to be provided in order to reduce the time needed to understand the main functionality of the software. Completing this process once does not necessarily free the developer from the task, it is a cycle that has to be repeated every time an update is introduced or a relevant part of the program has been changed. For a small single page application this might not sound dramatic, but considering complex applications developed by multiple teams, this task can raise the cost in time and resources as the information about the whole project has to be collected for the design of the documentation\cite{5712775}.

Software projects are bound by budget and deadline requirements. This leads to the fact that automation processes become essential where the reduction of the development time can be increased. It is therefore of great advantage that the generation of user documentation is modeled in such a way that it always reflects the most current development status of the software product. This thesis introduces the use of a graphical \gls{DSL} as an alternative way of modeling end user documentation.

Generation of user documentation should be an integral part of the development process. For the purpose of simplifying the layout of end user documentation this thesis proposes a solution for automatically generating documentation for the end users by using a DSL-Driven implementation approach.

\section{Contribution}
In this thesis we develope a CINCO-based\footnote{The “Cinco SCCE Meta Tooling Framework” is a generator-driven development environment for domain-specific graphical modeling tools. It is based on the Eclipse Modeling Framework and Graphiti Graphical Tooling Infrastructure, but aims to hide much of their complexity and intricate \glspl{API}. \url{https://cinco.scce.info/}} application for modeling different user action sequences in a web application and subsequently generating a markdown-based end user documentation. For simplicity's sake, we will focus on a web application of our own, the TODO-App. Nonetheless, the applied method can be extrapolate to any other web application, since the elementary building blocks of those applications are the same. The code for the application can be found here \url{https://github.com/MukendiMputu/UserDocGenerator}. 
\section{Outline}