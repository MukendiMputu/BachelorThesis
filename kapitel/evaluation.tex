\chapter{Evaluation}\label{ch:eval}
The purpose of this chapter is to explain our approach and its results. First, we justify the decision for the chosen approach to the topic, as well as the challenges posed and the solutions found for them. A brief description of the design pattern used and the resulting benefits is also given. Subsequently, a minimal system is used to demonstrate the development of end-user documentation. In addition, we present the resulting application, how it can be created and launched. Finally, the overall results and any alternative approaches will be discussed.

\section{Method selection}\label{sec:meth}

Selenium is at the core of our methodology, as it is per se the default automated testing and manipulation framework for web application. Not only the \acrshort{api} documentation is thorough and exhaustive, due to the fact that it is open-source, but it also allows the manipulation of the majority of the browser engines through WebDrivers~\cite{sel}. In addition, it integrates very well in a java as a maven dependency, hence well-fitted for purposes. For now, it has to be note that we do not intend to use Selenium for testing our appliation, rather for automatically executing the tasks or steps the documentation developer would have to do in order to create screen captures. In that sense, the Selenium WebDriver simulates the end-user steps and saves them as pictures in a dedicated folder.

Another state-of-the-art framework we chose is VuePress; it is well suited for creating static websites and is therefore perfect for designing technical documentation pages. Moreover, the effort required to configure a VuePress project is so low that the website can be up and running quickly. Alternatively, you can effortlessly bind the generated website to an existing domain, for example a company wiki page or similar. As for Selenium, VuePress also has a great online documentation page, with a step-by-step guide on how to quickly setup a project and lauch the server.

The link between the Selenium WebDriver and the VuePress project is the generated java maven project. Here, we decided to relieve the documentation developer of the task of adjusting the project properies, adding the dependencies, naming the different packages and so on. This are the taks an experienced java programmer would and as we intend to empower non-programmer to be confortable using our application, it is best we determine for ourselves implementation of said appliation.

Consequently, our editor application offered the capability to create a graphical model, which upon clicking the \textit{generate button} triggered the creation of both the java application and the VuePress project structure, and subsequently, the Selenium WebDriver took the indicated screenshots and saved them within the VuePress project folders.

\section{Setup}\label{sec:setup}

Since our editor appliation is a CINCO editor, which is based on Eclipse, setting up the development environment is straight forward --- at least if a certain acquintance with the Eclipse environment already exists. Beforehand, a java version must be present of the system in order to run the application. Also, to automate a web browser of choice, the corresponding Selenium WebDriver executable has to be downloaded and its location path has to be added to the system's \lstinline{PATH} variable. Concerning the Selenium libraries, we already take care of it by generation the pom.xml file, containing the required dependencies, along with the whole application structure.

After downloading and extracting the application package, it has to be started just like a common Eclipse IDE would normally be. A splash screen presenting the application and indicating the progress of the launch process appears. And within a few second a CINCO product IDE is started. 

\section{Results}\label{sec:res}

\section{Discussion}\label{sec:disc}

- Synchronisation issue with Selenium WebDriver,\\
- Checks (Training wheel protocol)\\
- Selenium screenshots at the right moment
