\chapter{Evaluation}\label{ch:eval}
The purpose of this chapter is to explain our approach and its results. First, we justify the decision for the chosen approach to the topic, as well as the challenges posed, and the solutions found for them. A brief description of the design pattern used, and the resulting benefits is also given. Subsequently, a minimal system is used to demonstrate the development of end user documentation. In addition, we present the resulting application, how it can be created and launched. Finally, the overall results and any alternative approaches will be discussed.

\section{Method selection}\label{sec:meth}

Selenium is at the core of our methodology, as it is per se the default automated testing and manipulation framework for Web application. Not only the documentation of the \gls*{api} is thorough and exhaustive -- since it is open-source -- it also allows the manipulation of most  of the browser engines through WebDrivers~\cite{sel}. In addition to that, it integrates very well in a Java project  as a Maven dependency, making it well-fitted for purposes. It must be noted that we do not intend to use Selenium for testing our application, rather than for automatically replicating the steps the documentation developer would have to take to create screen captures. In that sense, the Selenium WebDriver simulates the end user steps, captures the intermediate states of the Web application, and saves them as pictures in a dedicated folder.

Another state-of-the-art framework we chose is VuePress; it is well suited for creating static websites and is therefore perfect for designing technical documentation pages. Moreover, the effort required to configure a VuePress project is so low that the website can be up and running quickly. Alternatively, you can effortlessly bind the generated website to an existing domain, for example a company wiki page or similar. VuePress also has a great online documentation page, with a step-by-step guide on how to quickly setup a project and launch the server.

The link between the Selenium WebDriver and the VuePress project is the generated Java-Maven project. Here, we decided to relieve the documentation developer of the task of adjusting the project properties, adding the dependencies, naming the different packages and so on. This are the tasks an experienced java programmer would have to complete; and as we intend to empower non-programmer to be comfortable using our application, it is best we take care of the implementation of said application ourselves. We have also decided to generate dummy picture to make it possible to create the documentation with having the execute the Selenium script. This brings the advantage that different kind of picture can be added in place of the dummy ones in case the script is never intended to be executed.

Consequently, our editor application offered the capability to create a graphical model, which upon clicking the \textit{generate button}, triggers the creation of both the Java application and the VuePress project structure, and subsequently, the Selenium WebDriver takes the indicated screenshots and saves them within the VuePress project folders.

\section{Setup}\label{sec:setup}

The WebDoc editor application is a \textsc{Cinco} editor, which is based on Eclipse. Setting up the development environment is straight forward -- at least if a certain acquaintance with the Eclipse environment already exists. Beforehand, a version of Java must be present on the system to run the application. Also, to automate a Web browser of choice, the corresponding Selenium WebDriver executable must be downloaded, and its location path must be added to the system's \lstinline{PATH} variable. Concerning the Selenium libraries, we already take care of it by generating the pom.xml file, containing the required dependencies, along with the whole application structure.

After downloading and extracting the application package, it must be started just like a common Eclipse IDE would normally be. A splash screen presenting the application and indicating the progress of the launch process appears and within a few second a WebDoc IDE is started. 

\section{Results}\label{sec:res}
