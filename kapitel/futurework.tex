\chapter{Future Work}\label{ch:futwork}

We have identified a couple of development opportunities for future improvements:
\begin{itemize}
    \item \textbf{Cross-referencing model graphs}\\
        Reusability dictates that we be able to integrate complete model graphs in others to avoid modeling sequences multiple times. In this thesis, we achieved it by using the PrimeReference feature offered by the \textsc{Cinco} framework. Since \textsc{Cinco} itself is still under active development, cross-referencing DocGraphModels inside other could be subject to major improvements in the future, enabling us for example to list all available graph models in the palette, where they would be visible to the WebDoc user and more intuitive to integrate to the currently edited model.
    \item \textbf{Implement more checks}\\
        We have implemented so far module checks that help build syntactically correct model graphs, by enforcing the constraints defined on the meta-specification level of our graphical DSL. In the future, we could implement more checks to assist the developer in validation some other aspects of the model. For example, we could validate that the string values for the Web element selectors have the correct syntax of CSS selectors or XPath selectors by validating them against a well-constructed regular expression.
    \item \textbf{Vary the way of selecting Web elements (by XPath or CSS)}\\
        The example mentioned in the previous point raises the concern of allowing the developer to address Web elements by XPath and/or CSS. We have implemented so far only the user of CSS selector to address them. Many development frameworks for Web application use dynamic CSS id attributes, which makes it challenging to use as selector for the WebDriver to find. This issue could as well be addressed in future version of our application.
    \item \textbf{Extend the list of usable Web elements}\\
        One last point issue that require our attention is the list of available Web elements to use while modeling. For those who are experienced in Web development, it is not surprising that new HTML elements be implement in the future. We fast-developing technologies and the growing number of devices that can launch browser application, we might have to adapt and offer more model elements to reflect new ones. In the same way, we might be forced to remove some elements, since it is also possible that some HTML elements be marked as deprecated and not be rendered by any Web browser anymore.


\end{itemize}
