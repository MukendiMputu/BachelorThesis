% Chapter 1

\chapter{Mode-Driven Development with CINCO}\label{ch:Basis}

This chapter lays down the fundamentals of the \acrfull{mdd} using the CINCO SCCE Meta Tooling Framework, as well as the steps necessary to get up and running with the framework.
The term \acrshort{mdd} refers to software development paradigm where the functionalities of the software are first specified as models, from which then actual code can be generated.
To create and use models to represent the software system, a \acrfull{dsl} that is close to the problem domain, is needed. Furthermore, domain experts and programmers have to agree on how the specification language determines syntax and semantic of the model elements.

\section[CoreMDD]{Core Principles of \acrshort{mdd}}

The core principles of \acrshort{mdd} reside in the fact that software development is accelerated by providing a simple, but efficient abstractions of the software structure. Those model abstraction are often times followed by a automatic generation of the application code. Our work is to utilize the \acrshort{dsl} provide by the CINCO framework to design our graphical \acrshort{dsl}, which in turn will permit the generation of a functioning \gls{Selenium} Java application.

Oppeosed to the common development method, applying a graphical model to layout the different user sequences allows even non-programmer (here the domain expert with much more expertise on how to design a great software documentation) to accomplish the task of documenting the features offered by the web application. Nonetheless, the programmer has the tasks -- in collaboration wiht the domain expert -- to specify the meaning of each model element for the code generation process.

When applied correcty, the result of the model-driven development process is a tailored application to domain. This reflects one of the main advantages of \acrshort{mdd}, the accuracy of targeting directy the specific problem. Besides, it is still possible to change the \acrshort{dsl} so that it adapts to the new challenges emerging during the development process. This can be iterated until the specification reaches preciseness wanted to solve the problem.

\section[DSL]{Domain Specific Language}

A \acrfull{dsl}, as the name suggests, is a language adapted to specific development domain.