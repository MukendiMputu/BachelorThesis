% Chapter3.tex
\chapter{Cinco Product Application}\label{ch:CP}
Having described the meta-level of the user documentation model, we come now to the description of the graphical model instantiated from it. The goal of this chapter is to explain the the appearance and behavior of the various model elements. First, we give a strict definition of a graphical DSL, then illustrate the fundamental building blocks of our documentation model. Later on, we demonstrate the use of these graphical elements to specify the important configuration of the website that we want to document. At the end, we show how using the editor built-in generator, a project structure, that constitute the target application, is generated.

\section{Graphical DSL}\label{sec:gDSL}

Under \acrfullpl{dsl} we understand a languages tailored to describe or solve problems in a specific computational domain. They represent the core concept of most state of the art programming paradigms~\cite{perez-et_al}. As stated in~\cite{Naujokat2018}, one of their great advantage is that they permit domain experts with no programming experience to design application by means of graphical components, whose behavior and semantic meaning on the other hand have been or will be programed by developers with coding experience.

As a matter of fact, \acrshortpl{dsl} can be used in any thinkable domain because instances of a DSL are models and those can be used to represent any real-live object with all its possible interaction in a given environment.

\section{Model Graphs}\label{sec:ModElem}


\subsection{User Action Model Elements}\label{sec:FuncElem}


\subsection{Configuration Elements}\label{sec:ConfElem}


\section{Generation Process}\label{sec:GenProcess}

