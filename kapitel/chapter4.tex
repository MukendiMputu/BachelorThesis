% chapter4.tex
\chapter{End-User Documentation}\label{ch:userDoc}
The value of a good software product, in fact, of any product destined to be brought to the consumer is determined on how effective the end-user is able to use the product. Hence, putting a great effort to generate and manage a useful, well-structured documentation is as much important as the development of the product itself.

End-User documentation, also referred to as user documentation aims to provide the end users with the information necessary to properly interact with the software. It is part of the development life cycle and the bridge between the product developer's idea and the user. By improving the structure and usefulness of the information delivered to end-user, the developer not only reduces significantly the return of calls for support, but enhances also reputation of the product as well as of the producing company\cite{8584518}.

In this chapter we will go trough the main characteristics of end-user documentation, focusing on the standards establish by the ISO/IEC/IEEE.\@ Next we will talk about the specifics of documenting web application by giving an example of a documented web app, the TODO-App. Furthermore we will present the actually used methods for creating and managing such documentation.

%Types of End-User Documentation
\section{Characteristics}\label{sec:char}
There are many ways of characterizing an end-user documentation.
\section{Markdown}\label{sec:MD}
\section{VuePress}\label{sec:VP}