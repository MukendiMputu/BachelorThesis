\chapter{Final Remarks}\label{ch:epilogue}

\section{Conclusion}\label{sec:concl}

We implemented a \glsentryfull{webdoc} that creates an end-user documentation based on graph models using a graphical DSL. The use of graphical elements resembling HTML is advantageous because the application is easy to use and building model graphs occurs almost intuitively. Also, we have added the implementation of module checks to assist the designer in constructing a correct model graph that will generate executable application code.

Making use of the \textsc{Cinco} \glsentryfull{mgl} and \glsentryfull{msl}, we determined the look of each model element of our graphical DSL by keeping a close fidelity to the actual HTML elements they represent. That allowed us to create an editor application that enables the documentation designer to use those model elements to create graph diagrams illustrating the needed user workflow. Furthermore, by applying model checking methods to validate resulting diagrams, we ensure that generated executable application code be within the corresponding folder structure following established conventions.

We have been able to identify the characteristics of an end user documentation by following known standards and wisely applying the appropriate format to structure our documentation. As a result, we chose VuePress as the proper framework technology to transform it into a static website rendered in the Web browser.

The solution proposed in this thesis yields a well-structured and fully functional documentation website. Our approach uses the Selenium WebDriver, which navigates the Web application in a short amount of time while taking screen captures at the indicated places. Writing such a Selenium script would require deep knowledge of the Web automation framework and good programming skills in Java. In our solution, the WebDoc takes care of it for the developer.

The documentation developer would also benefit because the WebDoc conveniently creates the VuePress project with all the configuration files ready to go. This helps save time since a tutorial on how to create such a project is not necessarily needed.

\section{Future Work}\label{sec:futwork}

Web applications appears in many different forms. They can appear as a static page applications, or as complex constructs with many interfaces and enormous navigational graph. The results for a simple Web application, such as the TODO-App, were satisfactory to show the feasibility of our approach. A pontentila growth of the capabilities of the WebDoc lies in testing the other extremum of the type of Web applications we could document with it. By levaraging the capabilities of modern Web browser, i.e., to display responsive websites, we can improve the way of executing the Selenium script to obtain screenshots even for mobile version of the same applications. 

In the evaluation of the thesis, we have indentified a couple of development opportunities for future improvements:
\begin{itemize}
    \item \textbf{Start configuration}\\
        When creating a new documentation project using the project wizard, the user gets a empty project folder. We could ameliorate the creation process by generating a project template containing a start configuration. That is, a single .feat file containing an example graph as starting point of the documentation model. Such an example graph can jump
    \item \textbf{Cross-referencing model graphs}\\
        Reusability dictates that we integrate complete model graphs in others to avoid modeling sequences multiple times. In this thesis, we achieved it by using the PrimeReference feature offered by the \textsc{Cinco} framework. However, cross-referencing DocGraphModels inside others could be improved in the future to allow us, for example, to list all available graph models in a separate view. Thus, they would be visible to the WebDoc user and more intuitive to integrate into the currently edited model.
    \item \textbf{Implement more checks}\\
        We have implemented module checks that help build syntactically correct model graphs by enforcing the constraints defined on the meta-specification level of our graphical DSL. In the future, we could implement more checks to assist the developer in validating some other aspects of the model. For example, we could validate that the string values for the Web element selectors have the correct syntax of CSS selectors or XPath selectors by validating them against a well-constructed regular expression.
    \item \textbf{Language extension}\\
        The example mentioned in the previous point raises the concern of allowing the developer to address Web elements by XPath or CSS. So far, we have implemented only the use of CSS selectors to address them. However, many development frameworks for Web applications use dynamic CSS id attributes, making it challenging to use as a selector for the WebDriver to find. 

        One last point that requires our attention is the list of available Web elements to use while modeling. For those experienced in Web development, it is not surprising that new HTML elements appear in the future. In addition, with fast-developing technologies and the growing number of devices that can launch browser applications, we might have to adapt and offer more model elements to reflect new ones.
\end{itemize}
