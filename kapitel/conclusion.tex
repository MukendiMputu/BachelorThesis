\chapter{Conclusion}\label{ch:concl}

We have proposed a solution for a DSL-driven generation of end user documentation for Web applications. We presented our tasks management Web application as ongoing example of  using the \textsc{Cinco} Meta Tooling Suite, which has provided us with the possibility to implement a specification for our own graphical DSL.

Making use of the \textsc{Cinco} \glsentryfull{mgl} and \glsentryfull{msl}, we determined the look of each model element of our graphical DSL by keeping a close fidelity to the actual HTML elements they represent. This allowed us to create an editor application, that enables the documentation designer to use those model elements to create graph diagram illustrating the needed user workflow. By applying model checking methods to validate resulting diagrams, we ensure that executable application code be generated within corresponding folder structure following established conventions.

We have been able to identify the characteristics of an end user documentation by following known standards and wisely apply the appropriate format, namely Markdown, to structure our documentation and choose VuePress as the right framework technology to transform it to a static website rendered in the Web browser.