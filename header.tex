% header.tex

\documentclass[a4paper,11pt,twoside,english]{book}
\usepackage[a4paper,left=3.5cm,right=2.5cm,bottom=3.5cm,top=3cm]{geometry}

\usepackage[english]{babel}

\usepackage[pdftex]{graphicx,xcolor}
\usepackage{amsmath,amssymb}

% Bibtex german
\usepackage{bibgerm}

% URLs
\usepackage{url}
\usepackage{hyperref}
\hypersetup{
    colorlinks=true,
    linkcolor=blue,
    linktocpage=true,
    filecolor=magenta,     
    citecolor=green,
    urlcolor=magenta,
    pdftitle={DSL-Driven Generation of User Documentation}, 
    pdfauthor={Mukendi Mputu}
}
\urlstyle{same}

% Figure packages
\usepackage{subcaption}
\usepackage{multirow}

% List of Abbreviation
\usepackage[xindy, nonumberlist, acronym, toc]{glossaries}
% Glossary.tex
\makeglossaries{}

\newglossaryentry{Selenium}{
    name={Selenium},
    description={Selenium is a suite of tools for automating web browsers.}
}
    
\newglossaryentry{VuePress}{
    name={VuePress},
    description={VuePress is a Static Site Generator that generates pre-rendered static HTML for each page, and runs as an SPA once a page is loaded.}
}
        
% Acronyms
\newacronym{dsl}{DSL}{Domain-Specific Language}
\newacronym{cpd}{CPD}{Cinco Product Definition}
\newacronym{mgl}{MGL}{Meta Graph Language}
\newacronym{msl}{MSL}{Meta Style Language}
\newacronym{pom}{POM}{Page Object Model}
\newacronym{spa}{SPA}{Single Page Application}
\newacronym{api}{API}{Application Programming Interface}
\newacronym{html}{HTML}{Hypertext Markup Language}
\newacronym{iso}{ISO}{International Standards Organization}
\newacronym{iec}{IEC}{Third abbreviation}
\newacronym{ieee}{IEEE}{Institute of Electrical and Electronics Engineers}

\glsaddall{}

% Theorem environment
\usepackage[amsmath,thmmarks]{ntheorem}

% Correct display of umlauts
\usepackage[utf8]{inputenc}
\usepackage[T1]{fontenc}

% Algorithms
\usepackage[plain,chapter]{algorithm}
\usepackage{algorithmic}

\usepackage{enumerate}

% Using listings to highlight code
\usepackage{pxfonts}
\usepackage{listings}
\lstdefinelanguage{MGL}{
  sensitive = true,
  otherkeywords={},
  keywords = [1]{package, class, new, implements, override, CincoProduct, mgl, splashScreen, progressBar, progressMessage, image128, image16, image32, image48, image64, linuxIcon, about, plugins, prime, enum, for, import, entity, path, id, stylePath, graphModel, iconPath, diagramExtension, containableElements, attr, EString, EInt, EBoolean, as, style, extends, node, incomingEdges, outgoingEdges, container, edge, appearance, lineWidth, background, foreground, font, nodeStyle, roundedRectangle, rectangle, position, size, corner, text, value, BOLD, MIDDLE, CENTER, TOP, LEFT, appearanceProvider, image, multiText, text, lineStyle, transparency, DASH, DOT, edgeStyle, decorator, location, ARROW},
  frame=tb,
  framesep=0.7em,
  %framextopmargin=5em,
  numbers=left,
  stepnumber=1,                   
  numbersep=-10pt,
  showstringspaces=false,
  breaklines=true,
  comment=[l]{//},
  morecomment=[s]{/*}{*/},
  morestring=[b]',
  morestring=[b]"
}
\definecolor{codegreen}{rgb}{0,0.6,0}
\definecolor{codeblue}{rgb}{0,0,0.6}
\definecolor{codegray}{rgb}{0.3,0.3,0.3}
\definecolor{codepurple}{rgb}{0.5,0,0.33}
\definecolor{backcolour}{rgb}{0.95,0.95,0.95}
\lstdefinestyle{mystyle}{
    backgroundcolor=\color{backcolour},   
    commentstyle=\color{codegreen},
    keywordstyle=\color{magenta}\bfseries,
    numberstyle=\tiny\color{codegray},
    stringstyle=\color{codeblue},
    basicstyle=\ttfamily\linespread{1}\footnotesize,
    breakatwhitespace=false,
    captionpos=b,                    
    keepspaces=true,                
    showspaces=false,                
    showstringspaces=false,
    showtabs=false,                  
    tabsize=2
}
\lstset{style=mystyle}

% Caption Packet
\usepackage[margin=0pt,font=small,labelfont=bf]{caption}

% Set outline
%\setcounter{secnumdepth}{5}
%\setcounter{tocdepth}{5}

% Theorem options %
\theoremseparator{.}
\theoremstyle{change}
\newtheorem{theorem}{Theorem}[section]
\newtheorem{satz}[theorem]{Satz}
\newtheorem{lemma}[theorem]{Lemma}
\newtheorem{korollar}[theorem]{Korollar}
\newtheorem{proposition}[theorem]{Proposition}
% Without numbering
\theoremstyle{nonumberplain}
\renewtheorem{theorem*}{Theorem}
\renewtheorem{satz*}{Satz}
\renewtheorem{lemma*}{Lemma}
\renewtheorem{korollar*}{Korollar}
\renewtheorem{proposition*}{Proposition}
% Definitions with \upshape
\theorembodyfont{\upshape}
\theoremstyle{change}
\newtheorem{definition}[theorem]{Definition}
\theoremstyle{nonumberplain}
\renewtheorem{definition*}{Definition}
% Cursive font
\theoremheaderfont{}
\newtheorem{notation}{Notation}
\newtheorem{konvention}{Konvention}
\newtheorem{bezeichnung}{Bezeichnung}
\theoremsymbol{\ensuremath{\Box}}
\newtheorem{beweis}{Beweis}
\theoremsymbol{}
\theoremstyle{change}
\theoremheaderfont{\bfseries}
\newtheorem{bemerkung}[theorem]{Bemerkung}
\newtheorem{beobachtung}[theorem]{Beobachtung}
\newtheorem{beispiel}[theorem]{Beispiel}
\newtheorem{problem}{Problem}
\theoremstyle{nonumberplain}
\renewtheorem{bemerkung*}{Bemerkung}
\renewtheorem{beispiel*}{Beispiel}
\renewtheorem{problem*}{Problem}

% Customize algorithms %
\renewcommand{\algorithmicrequire}{\textit{Eingabe:}}
\renewcommand{\algorithmicensure}{\textit{Ausgabe:}}
\floatname{algorithm}{Algorithmus}
%\renewcommand{\listalgorithmname}{Algorithmenverzeichnis}
\renewcommand{\algorithmiccomment}[1]{\color{grau}{// #1}}

% Set line spacing %
\renewcommand{\baselinestretch}{1.25}
% Set floating environment %
\renewcommand{\topfraction}{0.9}
\renewcommand{\bottomfraction}{0.8}

% Blank page without page number, next page to the right
\newcommand{\blankpage}{
 \clearpage{\pagestyle{empty}\cleardoublepage}
}

% Kind of todo mark
\usepackage{xargs}                      % Use more than one optional parameter in a new commands
%\usepackage[pdftex,dvipsnames]{xcolor}  % Coloured text etc.
% 
\usepackage[colorinlistoftodos,prependcaption,textsize=tiny]{todonotes}
\newcommandx{\unsure}[2][1=]{\todo[linecolor=red,backgroundcolor=red!25,bordercolor=red,#1]{#2}}
\newcommandx{\change}[2][1=]{\todo[linecolor=blue,backgroundcolor=blue!25,bordercolor=blue,#1]{#2}}
\newcommandx{\info}[2][1=]{\todo[linecolor=OliveGreen,backgroundcolor=OliveGreen!25,bordercolor=OliveGreen,#1]{#2}}
\newcommandx{\improvement}[2][1=]{\todo[linecolor=orange,backgroundcolor=orange!25,bordercolor=orange,#1]{#2}}
\newcommandx{\thiswillnotshow}[2][1=]{\todo[disable,#1]{#2}}


% Image path
\graphicspath{{bilder/}}

% No single lines at the beginning of a section
\clubpenalty= 10000
% No single lines at the end of a section
\widowpenalty= 10000 \displaywidowpenalty= 10000
% EOF
